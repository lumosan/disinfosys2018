\subsection{Semantic Web} % (fold)
\label{sub:semantic_web}

\subsubsection{Semi-structured data} % (fold)
\label{ssub:semi_structured_data}
	Database schemas define fixed data structures for databases, labeling data values with names (\emph{well understood semantics}), and impose integrity constraints (for data consistency). This results in structurally and semantically coherent data.

	Structured layouts (e.g. HTML tables) provide a context, allowing to data values interpretation. Limitations of HTML include: no schemas, no constraints, data semantics difficult to analyse and share, structure of data expressed as layout.

	\textbf{Application-specific markup (tags)} are used to provide domain-specific meaning to the data in a document. The data that contains tags, markup, etc to specify the semantics of data values and to relate different data values is called \textbf{semi-structured data} (e.g. email, microformats, JSON, Extensible Markup Language XML).

	An XML can be viewed as (1) \emph{document model} (hierarchically nested tags enclosing textual content) and (2) \emph{data model} (semi-structured data model). This is good for exchanging data over communication networks.

	\textbf{Schema-less data} is more flexible and self-contained, but lacks consistency and certain optimizations are no longer feasible.
% subsubsection semi_structured_data (end)

\subsubsection{Semantic web} % (fold)
\label{ssub:semantic_web}
	User defined markup (\emph{schemas}) allows to share data interpretations across applications, resulting in \textbf{semantic hetereogeneity} (which is a problem for Web \emph{interoperability}). The W3C initiated the \emph{Semantic Web} initiative for this purpose, and it includes the XML framework.

	\emph{Semantic hetereogeneity} can be overcome by:
	\begin{itemize}
		\item Standardization -- agree on schemas, not useful for existing applications (solution a priori).
		\item Translation -- create mappings among existing schemas/DBs. Useful in small/controlled domains.
		\item Annotation -- reason over the conceptualization and reach an agreement for it (\emph{ontology)}.
	\end{itemize}
	Ways to relate entities together, according to ISO 14258 (Concepts and rules for enterprise models):
	\begin{itemize}
		\item Integrated approach -- common format for all models (\textbf{Standardization}).
		\item Federated approach -- no common format; for interoperability, accommodate on the fly. They must share a common ontology. (\textbf{Translation}).
		\item Unified approach -- there exists a common format but only at a meta-level, that provides smeantic equivalence to allow model mapping. (\textbf{Annotation}).
	\end{itemize}

	\textbf{Ontologies} are an explicit representation of a conceptualization of the real world. They should be encoded in a standardized form. They are created following either of these approaches:
	\begin{itemize}
		\item Ontology engineering -- manual effort; tools for editing and checking consistency.
		\item Automatic induction of ontologies -- from large document collections or existing structured resources.
	\end{itemize}

	Model requirements for ontologies are:
	\begin{itemize}
		\item \textbf{Simplicity}, to encourage wide-spread use. Some existing ontologies (e.g. Cyc) are expressed in fairly complex knowledge representation models.
		\item \textbf{Exchangeability}: any kind of data that is processed must be easily exchangeable. This motivated the use of XML as data representation format in the first place.
		\item \textbf{Non-intrusive annotation}: data interpretation is associated with data a-posteriori, and there may be multiple interpretations for the same data.  This excludes, for example, the approach to use XML elements for annotation.
		\item \textbf{Domain-specific vocabularies}: the model must provide a mechanism that allows to specify schemas for different domains.
		\item \textbf{Modeling primitives}: they need to offer a sufficiently rich set of possibilities to model complex situations.
		\item \textbf{Reasoning Capabilities}: even simple forms of reasoning within the ontology layer can make the interpretation of the data much more powerful (and thus the processing in the Semantic Web).
	\end{itemize}

	\emph{The Semantic Web standards RDF and OWL are positioned in the Semantic Web architecture in top of the syntactic layer}
% subsubsection semantic_web (end)

\subsubsection{Resource Description Framework (RDF)} % (fold)
\label{ssub:rdf_rdf_resource_description_framework}
	RDF consists of two parts:
	\begin{itemize}
		\item Language for representing metadata instances, which allows to annotate Web resources with statements (\textbf{RDF (instances)}). The web resources are addressed by \emph{Universal Resource Identifiers (URI)} (e.g. URLs).
		\item Language for specifying schemas for RDF instances (\textbf{RDF-schema}). This language enables specification of the vocabulary and grammar that is used for forming statements for annotation.
	\end{itemize}

	\emph{Since RDF
statements can be created also without using RDF schemas, RDF is a semi-structured data
model, similar as with well-formed XML (instances) and XML-DTD (schemas)}.

\emph{The RDF model is similar to the entity-relationship (ER) model. Entities correspond to
resources and relationships correspond to properties. The main difference is that RDF
requires that relationships are directed, and have a specific semantics: the resource from
which the (directed) relationship emerges is assigned a property with the value to which the
relationship points. }
% subsubsection rdf_rdf_resource_description_framework (end)

\subsubsection{Semantic web resources} % (fold)
\label{ssub:semantic_web_resources}

% subsubsection semantic_web_resources (end)

% subsection semantic_web (end)