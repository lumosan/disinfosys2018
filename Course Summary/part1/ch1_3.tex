%\todo[inline]{w1 slides 50-58}

\textbf{Information management tasks}:
\begin{itemize}
	\item Model $\leftrightarrow$ Data:
	\begin{itemize}
		\item \textbf{Retrieval}: given a model, find some data
		\item \textbf{Data mining}: given data, find a model for it. Creates higher level abstractions from lower level data. Uses statistical and ML methods, rule-based approaches, typically large data set. Also called data science / data analytics.
	\end{itemize}
	\item Model $\leftrightarrow$ Real world:
	\begin{itemize}
		\item \textbf{Conceptual modeling}: Analyze the real world and specify a model
		\item \textbf{Evaluation}: Given a model, evaluate it against reality.
	\end{itemize}
	\item Data $\leftrightarrow$ Real world:
	\begin{itemize}
		\item \textbf{Control}: (\emph{Output} -- data visualization; control.)
		\item \textbf{Monitoring}: (\emph{Input} -- users; sensors.)
	\end{itemize}
	\item Interoperability:
	\begin{itemize}
		\item \textbf{Semantic}: Between information systems that implement a model.
		\item \textbf{Syntactic}: Between data stored in different ways.
	\end{itemize}
\end{itemize}

    \begin{figure}[htp]
      \centering
        \includegraphics[width=\textwidth]{images/infotasks.png}
        %\missingfigure[figwidth=\textwidth]{Some Figure}
        \caption{Information Management Tasks}
        \label{fig:infotasks}
    \end{figure}

\textbf{Purpose of an Information System} -- Users need it to take decisions. The \textbf{utility of information} is linked to the value achieved. The \textbf{value} depends on \emph{importance} and on \emph{quality} of the decision. Quality of decision depends on quality and understandability of information.

    \begin{figure}[htp]
      \centering
        \includegraphics[width=\textwidth]{images/infosysagain.png}
        %\missingfigure[figwidth=\textwidth]{Some Figure}
        \caption{Refined view of an IS}
        \label{fig:infosysagain}
    \end{figure}