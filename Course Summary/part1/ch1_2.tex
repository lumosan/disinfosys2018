%\todo[inline]{w1 slides 20-48}
A \textbf{model M} is represented using a \textbf{data model D}. A data model D uses data structures and operations for representing constants, data, functions and constraints of a model M within a computer system.\\
Examples:
\begin{itemize}
	\item \textbf{Associative array}: \emph{abstract data structure} (abstract data type, i.e. different physical implementations are possible) used to represent \emph{functions}. Manages a set of keyvalue pairs (representing the function) and supports set of operations to manipulate the associative array.
	\item Labelled graphs
	\item Relational tables
\end{itemize}

The collection of data represented in a data model D is called a \textbf{database DB}.\\
\textbf{Database management system (DBMS)}: Computer system designed to manage databases and thus realize information systems (but many information systems use databases without a database system). It implements a \textbf{data modeling formalism} and are application (or domain) independent. The data is stored using different \textbf{encodings} of data structures exploiting available \textbf{storage} media, to manage large DBs efficiently.

\textbf{Data model}: (1) \emph{data model used to represent a model within a computer system} (this is the sense we will use in the following); (2) formalism \emph{language} to specify a whole class of data models, consisting of \emph{Data Definition Language (DDL, to define a table with a specific structure)} and \emph{Data Manipulation Language (DML, to formulate queries)}.\\
The specification of a data model using a DDL is called a \textbf{database schema} or \textbf{schema} S.

\textbf{Data modeling languages} specify three components:
\begin{itemize}
	\item Data structures -- describes of how constants and functions are represented as data structures (represent databases).
	\item Integrity constraints -- have to be respected by the facts and can be expressed using a specific language.
	\item Manipulation -- data manipulation operators enable manipulation of the databases (update, transform...).
\end{itemize}

\textbf{Levels of Abstraction}: levels (not strict) of increasing interpretation of raw data, obtained by automated and/or human analysis of the data.

    \begin{figure}[htp]
      \centering
        \includegraphics[width=.7\textwidth]{images/levabs.png}
        %\missingfigure[figwidth=\textwidth]{Some Figure}
        \caption{Levels of Abstraction}
        \label{fig:levabs}
    \end{figure}

\begin{itemize}
	\item \emph{Unstructured data} -- Captured from real world, through measurements and human input. Structure given by static data types. Little ``semantic'' meaning.
	\item \emph{Structured data} -- Data structured according to a ``static'' schema, implementing an interpretable model. E.g. relational data model.
	\item \emph{Knowledge} -- Schema can evolve dynamycally as knowledge expands. More applications. Inference can be used to derive more knowledge. More flexible.
\end{itemize}

\textbf{Logical data independence} -- \emph{An abstract representation
of data is used that is independent of the underlying
implementation of the data at the lower abstraction level}. The same data can be interpreted in different ways. \emph{Views} support different schemas for accesing the same data stored in a DB. Corresponds to supporting diff. models based on same data. \emph{To support evolution of schemas}.

\emph{An information system is (typically) based on a database management system. The information system is concerned with providing a model for a real-world aspect, whereas the database system is concerned with the efficient management of the data structures required to represent the model.}

\textbf{Key data management tasks}:
\begin{itemize}
	\item Efficient management of large amounts of data -- how the data is mapped to
	the different available storage media, and what auxiliary data
	structures, so-called indices, are created to support the efficient
	access.\\
	How operations are performed on the data.
	\item Ensure persistence (store data independently from programs lifetimes) and consistency (correct data independently from failures) of data through updates and failures.\\
	Transient data is data maintained independently of the lifetime of any program (i.e. persistent data).
	\item Exploit different media.
\end{itemize}

Row store $\rightarrow$ easy to add a tuple; Colum store $\rightarrow$ less expensive to read relevant data.

\textbf{B+-Trees}: Perform well for very large DBs and can be efficiently
managed on available storage media. They are designed s.t.:
\begin{itemize}
	\item Tree nodes can be stored efficiently on the available disk blocks (e.g. by having a high fanout in the tree).
	\item the tree remains balanced independent of how the data (more precisely the search keys) are distributed, in order to guarantee a low depth of the tree and thus efficient search and log. time update operations.
\end{itemize}
In order to guarantee these properties,
updates to the B+-Tree might require additional operations to
restructure the tree -- e.g. add a new level in the tree after insertion (see Figure \ref{fig:ch1tree1}).
    \begin{figure}[htp]
      \centering
        \includegraphics[width=\textwidth]{images/ch1tree1.png}
        %\missingfigure[figwidth=\textwidth]{Some Figure}
        \caption{Levels of Abstraction}
        \label{fig:ch1tree1}
    \end{figure}

\newpage

\textbf{Issues to be taken care of:}
\begin{itemize}
	\item \textbf{Optimizing access to DBs}:
	\begin{itemize}
		\item \textbf{Physical DB design} -- Is done \textbf{before} accesses to DB. Choose index and storage to match applications requirements, depending on applications':
		\begin{itemize}
			\item Different types of accesses
			\item Different access patterns
		\end{itemize}
		\item \textbf{Declarative Query Optimization} -- Is done \textbf{at the time} the access to the DB is executed. For a logical operation (choice of best algorithm, given storage and indexing scheme, when a query is to be executed).
	\end{itemize}
	\item \textbf{Safe storage and updates}:
	\begin{itemize}
		\item Support of consistent state with:
		\begin{itemize}
			\item Multiple users (\textbf{concurrency}) $\rightarrow$ \textbf{Isolation} (users do not affect each other's transactions).
			\item Presence of failures (\textbf{recovery}) $\rightarrow$ \textbf{Durability} (after executing a transaction, the result is never lost) and \textbf{atomicity} (transactions are either completely or not at all executed).
		\end{itemize}
	\end{itemize}
\end{itemize}

\textbf{Physical data independence}\\
(Second central data independence concept in DBs, after logical data independence)\\
The same logical database can be physically realized in many different ways, without affecting the result. Logical operations (queries, updates) can be executed in
physically different ways (using declarative query optimization and transaction management).

    \begin{figure}[htp]
      \centering
        \includegraphics[width=\textwidth]{images/modarch.png}
        %\missingfigure[figwidth=\textwidth]{Some Figure}
        \caption{Modelling architecture -- distinction between application-specific models of information systems, data models used for their implementation in a database systems, and different physical realizations of data models (1975)}
        \label{fig:modarch}
    \end{figure}
    \newpage
    \begin{figure}[htp]
      \centering
        \includegraphics[width=\textwidth]{images/refview.png}
        %\missingfigure[figwidth=\textwidth]{Some Figure}
        \caption{Semantic layer (interpretation of the real world); syntactic layer (uses a database management system); physical layer (storage)}
        \label{fig:refview}
    \end{figure}

