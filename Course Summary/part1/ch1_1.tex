An information system is a software that manages a model of (some aspect of) the real world within a (distributed) computer system (for a given purpose).

\begin{itemize}
	\item \textbf{real world} -- anything from abstract concepts (e.g. a legal information system) to technical systems including computer system or networks itself (e.g. information systems for network management).
	\begin{itemize}
		\item \textbf{Physical phenomena}: measure environment and create models of physical phenomena (\emph{meteorological
		information systems})
		\item \textbf{Social organization}: capture roles, relationships, activities... in social organizations (\emph{finance}, \emph{logistics}...)
		\item \textbf{Human thought}: Model human thought and reasoning processes. They capture the meaning of text and other media, assess importance and quality of information, model human traits... (e.g. \emph{web search engines}).
	\end{itemize}
	\item \textbf{purpose} -- an information system has an entity (human, computer) that uses it to perform a certain task related to some aspect of the real world.
	\item \textbf{aspect} -- there are many ways to represent the real world and same aspects of the real world in information systems, depending on the purpose.
	\item \textbf{model} -- mathematical structure consisting of a set of constants, functions and axioms (constants and functions must be consistent with axioms). \\
	It is linked by an interpretation function (\emph{homomorphic}, bc. functions in the model preserve real wold relationships) to the real world. Difficult to check functions in real world -- indirect methods are required.\\
	Examples of formal models: Entity-Relationship models (derived from knowledge representation mechanisms developed in AI), OWL (generalization of entity-relationship model enabling logical inference for concept classes; basic model for the Semantic Web), Graph models (social network, biological network, and communications network data), Vector space models (represent feature spaces of text and media content), Probabilistic models (uncertainty in content and sensor data), Differential equations and simulation programs (behaviors of complex systems), Process models (capture structure and dynamics of business processes, also called workflows).
	\begin{itemize}
		\item Constants (identifiers) -- give names to real world things
		\item Functions (relations) -- relate objects with their properties and different objects among each other. Can be represented:
		\begin{itemize}
			\item Explicit representation (by enumeration) -- \textbf{data}
			\item Implicit representation (by specification or algorithm)
		\end{itemize}
		\item Axioms (constraints) -- state which properties are possible and which are not
	\end{itemize}
\end{itemize}