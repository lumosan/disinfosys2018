\subsection{Introduction to Data Mining} % (fold)
\label{sub:introduction_to_data_mining}
  Information systems create a model of reality based on data. \textbf{Key tasks}:
  \begin{itemize}
    \item \textbf{Data mining} (\emph{data} to \emph{models}) -- Given data, find a model that matches the data; provide algorithms that reveal hidden structures in data
    \item \textbf{Retrieval} (\emph{models} to \emph{data}) -- Given a model, find some specific data
  \end{itemize}

  \textbf{Growing data collections}: the rate at which digital data is produced now exceeds the rate at which storage space grows. Most data is useless, but \emph{data mining} (or \emph{data analytics}) allow to extract \textbf{actionable insights} (\emph{understandable} and \emph{useful}).

  \textbf{Challenges}:
  \begin{itemize}
    \item\emph{Practical}: Data access and ownership (legal and political reasons, different systems); Domain knowledge and expertise (what we're looking for)
    \item\emph{Technical}: Data management (Big Data); Data mining algorithms
  \end{itemize}

  \textbf{Classes of Data Mining Problems}:
  \begin{itemize}
    \item \emph{Local Properties} -- Patterns (association rules, pattern mining)
    \item \emph{Global model} -- Descriptive model (clustering, information retrieval) and Predictive model (classification, regression)
    \item \emph{Exploratory Data Analysis} -- used when no clear idea exists and as preprocessing
  \end{itemize}

  \textbf{Components of Data Mining algorithms}:
  \begin{enumerate}
    \item Pattern structure / Model representation (\emph{what we look for}), e.g. dependencies, clusters, decision trees
    \item Scoring function (\emph{how well the model fits the data set}), e.g. similarity functions
    \item Optimization (\emph{parameter tuning}) and search (\emph{find data satisfying a pattern})
    \item Data management (\emph{implement algorithm for large datasets}), e.g. use of inverted files
  \end{enumerate}

  \textbf{Data Mining System}: Data mining algorithms are part of larger data mining system that support the pre- an post-processing of the data. A data mining systems performs the following typical tasks (each step can influence the preceding steps):
  \begin{enumerate}
    \item Data extraction / integration / transformation / cleaning -- The integrated data is kept in \emph{data warehouses} (databases replicating and consolidating the data). Data cleaning consists on removing inconsistent and faulty data. Data integration and data cleaning are supported by \emph{data warehousing systems}.
    \item Data selection -- Subsets of the data can be selected from the \emph{data warehouse} for performing specific data mining tasks targeting a specific question. Task-specific data collections are called \emph{data-marts}.
    \item Data analytics -- The \emph{data mining} algorithm is applied to the \emph{data-mart}. Data mining detects patterns in the data (e.g. association rule mining).
    \item Pattern / model assessment -- Once specific patterns are detected they can be further processed, e.g. for evaluating how interesting the patterns are.
  \end{enumerate}
% subsection introduction_to_data_mining (end)

\subsection{Association Rule Mining} % (fold)
\label{sub:association_rule_mining}
  \subsubsection{Pattern structure} % (fold)
  \label{ssub:pattern_structure}
    \todo[inline]{w6 slides 21-24}
  % subsubsection pattern_structure (end)
  \subsubsection{Scoring function} % (fold)
  \label{ssub:scoring_function}
    \todo[inline]{w6 slides 25-28}
  % subsubsection scoring_function (end)
  \subsubsection{Building trees} % (fold)
  \label{ssub:building_trees}
    \todo[inline]{w6 slides 30-49}
  % subsubsection building_trees (end)
  \subsubsection{Sampling and partitioning} % (fold)
  \label{ssub:partitioning}
    \todo[inline]{w6 slides 50-57}
  % subsubsection partitioning (end)
  \subsubsection{FP Growth} % (fold)
  \label{ssub:fp_growth}
    \emph{Apriori is original and most popular rule mining algorithm but others provide better performance under certain circumstances}

    FP Growth is a frequent itemset discovery \emph{without} candidate itemset generation that aims at main memory implementations (thus less suitable for distributed implementation).

    \textbf{Steps}:
    \begin{enumerate}
      \item Build the FP-tree datastructure. It requires 2 passes over the dataset.
      \item Extract frequent itemsets directly from the FP-tree.
    \end{enumerate}

    \textbf{FP-Tree}
    \todo[inline]{w6 slides 58-72}
  % subsubsection fp_growth (end)
% subsection association_rule_mining (end)